%!TEX root = ../construction.tex
% -*- root: ../construction.tex -*-

건축 산업(Construction Industry)에 IT 기술을 적용하여 기존 건축 산업의 문제를 극복하고자 하는 연구들이 다양하게 진행되고 있다. 이러한 연구들이 해결하고자 하는 문제는, 첫번째로 건축 작업에서 원하는 정보를 적절하게 제공해줌으로써 정확한 건축물을 지을 수 있도록 하는 것이다. 이를 위해서는 실제 현장에서 획득하기 힘든 정보를 즉각적으로 제공하여야 하고\cite{yeh_-site_2012,cote_augmented_2013,chi_research_2013}, 2D 데이터 뿐만이 아니라 다양한 형태의 정보를 제공함으로써 현장 근로자가 적절한 정보를 제공받을 수 있도록 하여야 한다\cite{behzadan_ubiquitous_2008,schall_handheld_2009}. 

다음으로 해결하고자 하는 문제는, 건축 작업에 참여하는 다양한 관련자들(stakeholder) 사이의 효율적인 의사소통을 지원하고자 하는 것이다. 건축 작업에는 다양한 전문가가 참여하고, 다양한 문제들이 발생한다. 따라서, 효율적인 의사소통이 지원되지 않을 경우에 작업의 비용과 기간이 증가할 뿐만 아니라\cite{schall_handheld_2009}, 건축물의 안전성도 떨어지게 된다\cite{wagner_building_2012,lin_using_2014}. 따라서 IT 기술을 활용하여 관련자들 사이의 커뮤니케이션을 지원하여 빠르고 정확하게 의사결정을 이루는 것이 중요하다. % 본 논문에서는 이러한 문제점들을 해결하기 위하여 2D 데이터에서부터 건축물의 전체적인 3차원 모형, 내부 벽면에 숨겨진 정보까지 다양한 건축물의 정보를 동시에 제공하고 이를 수정할 수 있도록 하는 포터블 기기를 제안한다. 또한, 건축 현장에서의 커뮤니케이션을 지원하기 위하여 프로젝션 기술을 이용하여 구현함으로써 서로 다른 전문가들이 현장에서 참여하여 의사소통할 수 있도록 하였다.


% 첫번째 문제를 해결하기 위하여, 건축 현장에서 필요한 정보를 적절하게 제공하기 위한 다양한 연구가 이루어지고 있다. 이러한 연구는 제공되는 정보의 종류에 따라 2D data, 3D model, MEP(Mechanical, Electronical, and Plumbing) 기반의 연구로 구분할 수 있다\cite{ebbesen_information_2015}. 
이러한 문제점을 극복하기 위하여 건축현장에서 IT기술을 활용하는 다양한 연구들이 이루어지고 있다. 이 연구들은 제공되는 정보의 종류에 따라 2D Data, 3D Model, MEP 기반의 연구로 분류된다. 2D Data 정보의 경우, 도면 정보를 기반으로 상세한 건축 정보를 제공받을 수 있지만\cite{yeh_-site_2012,ishii_augmented_2002,cote_augmented_2013}, 3차원 모델 정보의 제공이 부족하므로 이를 해석하는 과정에서 Image Mismatch가 발생하기 쉽다. 이러한 문제점을 극복하기 위해 2D Data를 이용하여 3D 모델을 제공하는 연구들이 제시되었다\cite{wagner_building_2012,dong_collaborative_2013, hou_combining_2014,behzadan_ubiquitous_2008,williams_bim2mar:_2015}. 이러한 연구는 현실 환경과 같이 최종 건축물의 3차원 정보를 조망할 수 있다는 장점이 있으나, 건축 현장의 세부적인 정보를 획득하기에 어려움이 존재한다. 

최근에는 이러한 정보를 제공하기 위하여 Building Information Management(BIM) 정보를 기반으로 건축물 내의 감춰진 Mechanical, Electronic, and Plumbing (MEP) 구조물의 정보를 제공하는 연구가 진행되고 있다\cite{schall_handheld_2009,olbrich_augmented_2013,kwon_defect_2014,webster_augmented_????,golparvar-fard_d4ar4-dimensional_2009}. 이러한 연구는 건축물 부분의 세부 정보를 획득할 수 있으나, 전체적인 구조를 파악하기 어려웠다\cite{webster_augmented_????}. 본 논문에서는 효율적인 정보 제공을 위해 2D 데이터에서부터 건축물의 전체적인 3D 모델과 내부 벽면에 숨겨진 MEP 정보까지 전체적인 정보를 제공하고자 하였다. 특히, 이러한 다양한 정보를 종합적으로 제공하기 위하여 Multi-view 기반의 Interactive Screen 기술을 이용하여 이러한 정보들을 효율적으로 제공하는 시스템을 제안하였다

% 두번째 문제를 해결하기 위하여, IT 기술을 이용하여 다양한 관계자 사이의 효율적인 의사소통을 제공하기 위한 다양한 연구가 진행되고 있다\cite{ishii_augmented_2002,klein_imaged-based_2012,chi_development_2012,lin_using_2014}. 
이러한 연구들은 커뮤니케이션 환경에 따라 Desktop 환경에서 모바일 기기 환경, Projection 기반의 Interactive Surface 환경으로 진화하였다. 먼저, Desktop 기반의 연구\cite{dong_collaborative_2013,golparvar-fard_d4ar4-dimensional_2009,lin_using_2014}들은 고정적인 환경인 Off-Site에서 진행되며, 전체적인 조망이나 정보 공유가 가능하다. 이러한 연구들의 경우, Off-Site에서의 협업은 가능하나 시스템 구조상 On-Site 에서는 활용되기 어렵다는 단점이 있다. 이러한 단점을 극복하기 위해 On-Site에서 동작 가능한 모바일 기반의 연구들\cite{saidi_value_????,kwon_defect_2014,cote_augmented_2013,irizarry_ambient_2014,hammad_distributed_2009,bae_high-precision_2013,ammari_collaborative_2014,williams_bim2mar:_2015}이 진행되었다. 실제 건축 현장에서 Mobile Augmented Reality 기술을 이용하여 도면이나 3D 모델 등의 정보를 확인 할 수 있고, 필요에 따라 수정 가능하다. 하지만 모바일 기기의 특성상 화면 공유보다는 개인을 위한 Personal Workspace를 제공하기 때문에 협업에 제한적이다. 

이러한 문제를 극복하기 위해, Desktop과 Mobile 기반 연구들의 단점을 보완하여 On-Site 에서 건축 관계자들의 협업을 지원하기 위해 Projection 기반의 Interactive surface 기반의 연구들\cite{ishii_augmented_2002,wagner_building_2012,song_penlight:_2009}이 제시되고 있다. 이러한 기술은 Projection 기술을 이용하여 Interaction Space를 제공하기 때문에 여러 사용자들이 참여하여 협업하기에 적합하다. 따라서, 현장에서의 효율적인 의사소통을 위해서는, 개인화 된 기기 보다는 여러 사용자의 참여가 가능한 Projection 기반의 Interactive Surface 기술을 이용하는 것이 적합하다.




%이러한 문제를 극복하기 위하여 건축 관계자들 간의 협업을 위해 다양한 연구들이 진행되고 있다. 먼저, Desktop 기반의 연구[3, 15, 21]들은 고정적인 환경인 Off-Site에서 진행되며, 전체적인 조망이나 정보 공유가 가능하다. 이러한 연구들의 경우, Off-Site 에서의 협업은 가능하나 시스템 구조상 On-Site에서는 활용되기 어렵다는 단점이 있다. 이러한 단점을 극복하기 위해 On-Site에서 동작 가능한 모바일 기반의 연구들[2, 11, 12, 17-19]이 진행되었다. 실제 건축 현장에서 Mobile Augmented Reality 기술을 이용하여 도면이나 3D 모델 등의 정보를 확인 할 수 있고, 필요에 따라 수정 가능하다. 하지만 모바일 기기의 특성상 화면 공유보다는 개인을 위한 Personal Workspace를 제공하기 때문에 협업에 제한적이다. 이러한 문제를 극복하기 위해, Desktop과  Mobile 기반 연구들의 단점을 보완하여 On-Site에서 건축 관계자들의 협업을 지원하기 위해 Projection 기반의 Interactive surface 기반의 연구들[4, 8, 10, 20]이 제시되고 있다. 이러한 기술은 Projection 기술을 이용하여 Interaction Space를 제공하기 때문에 여러 사용자들이 참여하여 협업하기에 적합하다. 따라서, 현장에서의 효율적인 의사소통을 위해서는, 개인화 된 기기 보다는 여러 사용자의 참여가 가능한 Projection 기반의 Interactive Surface 기술을 이용하는 것이 적합하다.

% 또한, 시스템적으로 가상현실 기술 기반, 모바일 어플리케이션 기반, 증강현실 기술 기반, Interactive Surface 기반의 연구들이 진행되고 있다. 가상현실 기술 기반의 연구들\cite{lin_using_2014,dong_collaborative_2013}들은 현장보다는 사전 준비 단계에서 활용되며, 전체적인 조망이 가능하다는 장점이 있으나, 시스템 구조 상 on-site에서는 활용되기 어렵다는 문제가 있다. xxx

%%%%%%


% 본 논문은 이러한 문제점을 극복하고 건축 현장에서 실제로 활용이 가능한 시스템을 제공하기 위하여 Interactive Surface 기술\cite{grossman__2010} 기반의 2D/3D 건축 정보 Visualization System을 제안한다. 특히, 건축 현장에서 활용도를 높이기 위하여 모바일 프로젝터와 깊이 인식 카메라를 이용하여 portable 시스템을 구현하였다. 기존 Interactive Surface 기술과 다르게 2D/3D 정보를 동시에 보여주기 위하여 Multiscreen Interactive Surface 기술을 적용하였으며, Portable 하게 Multiscreen을 구현하기 위하여 Image Marker 기반의 Calibration 기술을 개발하여 미리 설치된 환경이 아니라 실시간으로 Multiscreen을 구성하도록 하였다. 이를 이용하여 현장의 'L'-shape 벽면에 걸쳐 프로젝션함으로써 2D Interaction 을 위한 Horizontal Screen과 3D Interaction을 위한 Vertical Screen을 구성하였다. 이를 통하여 건설 현장에서도 쉽고 빠르게 건축 정보에 접근함으로써 현장에서의 협업을 높일 수 있었다. 

% 하면서 실시간으로 건축 모델의 정보를 업데이트하고, 작업 중에도 편리하게 협업할 수 있는 시스템을 설계하였다. 본 논문에서는 이러한 시스템을 이용하여 건축 현장에 적용하기 위한 design을 수행하고, 이를 건축 관련자들에게 사용 후 informal user study와 비교 연구를 적용하여 시스템의 유용성을 검증하였다. 추가적으로 실험의 피드백과 구현 상의 문제들을 정리하고 결론을 도출하였다.
% 논문의 구성은 2장에서 제안하는 시스템의 구조와 HW, SW 설계 및 구현, 상호작용 설계에 대하여 설명한다. 3장에서는 사용자 대상의 실험 설계 및 결과에 대하여 서술하고, 4장에서 실험 결과에 대한 discussion을 기술한다. 그리고 5장에서 결론을 짓도록 한다.

본 논문에서는 건축 작업 시 현장에서 필요한 2차원 건축 정보부터 3차원 모델 뷰와 숨겨진 MEP 정보까지, 다양한 정보가 동시에 필요한 문제를 해결하고, 이를 통하여 현장 근로자 사이의 효율적인 의사소통 지원함으로써 협업 문제를 해결하고자 한다. 이를 위해, 건축 현장에서 활용 가능한 Portable projection 기반 Interactive Surface 기술을 이용하여 건축 정보 시스템을 구성하였다. 본 연구에서 제안하는 시스템은 효율적인 협업을 위해 프로젝터를 이용하여 Shared workspace을 제공하여 현장의 여러 사용자들이 참여가 가능하도록 설계하였고, 2D 도면의 정보와 3D 모델의 정보를 동시에 제공함으로써 정보 접근의 효율성을 높였다. 2D 도면정보와 3D 모델 정보를 동시에 제공하기 위하여, multi-screen 기반 interactive surface 기술\cite{coram_astrotouch:_2013,weiss_benddesk:_2010,wimmer_curve:_2010,benko_miragetable:_2012}을 구현하였다. 

multi-screen 기반 interactive surface기술의 stationary한 기존 기술의 한계를 극복하기 위하여 마커 기반의 실시간 calibration 기술을 개발하였다. 이렇게 구성된 Interactive Surface 환경에서 직관적인 건축 정보의 제어를 위해 Natural User Interface(NUI) 기술을 이용하여 상호작용 하도록 하였다. 이는 seamless하게 건축 정보에 접근하면서 실시간으로 건축 모델의 정보를 업데이트하며, 편리하게 조작이 가능하다. 
본 논문에서는 이러한 시스템을 건축 현장에 적용하고, 실제 건축 관계자들을 대상으로  informal user study와 비교 연구를 진행하여 시스템의 유용성을 검증하였다. 이러한 실험을 기반으로 피드백과 구현상의 보완을 진행하여 결론을 도출하였다. \textit{실험 결과 추가!}

논문의 구성은 2장에서 제안하는 시스템의 구조와 HW, SW 설계 및 구현, 상호작용 설계에 대하여 설명한다. 3장에서는 사용자 대상의 실험 설계 및 결과에 대하여 서술하고, 4장에서 실험 결과에 대한 discussion을 기술한다. 그리고 5장에서 결론을 짓도록 한다.



