%!TEX root = ../Sensors_SmartConstruction.tex
% -*- root: ../Sensors_SmartConstruction.tex -*-

IT 기술을 이용하여 기존 건축 환경의 문제를 극복하고자 하는 연구들이 다양하게 진행되고 있다. 이러한 연구들에서 해결하고자 하는 기존 건축 환경의 문제점은 먼저, 현장에서 필요한 정보에 즉시 접근하기가 어렵다는 점이다\cite{yeh_-site_2012,cote_augmented_2013,chi_research_2013}. 기존의 건축 환경에서는 종이 기반의 건축 도면(paper plan)을 이용하여 정보를 획득하였다. 이러한 종이 건축 도면은 프로젝트 전반의 다양한 분야(various construction disciplines)에 대한 정보를 포함하기 때문에 부피가 크고 이용하기 불편한 문제가 있다\cite{yeh_-site_2012}. 또한, 2D 이미지에서 3D 모델을 유추하는 과정에서 이용자 사이에 Image Mismatch가 발생하기 쉬우며\cite{cote_augmented_2013,yeh_-site_2012}, 약속된 기호(predefined symbol)을 사용하기 때문에 이를 해석하기 위한 인지적 부하(cognitive load)가 발생하는 문제\cite{chi_research_2013}가 있어 현장에서 활용이 떨어지고, 정보 활용이 부족해지는 문제가 있었다. 

이러한 문제점을 극복하고 현장에서 편리하게 정보를 제공하기 위하여 건축 정보의 Visualization에 대한 연구가 이루어지고 있다. 이러한 연구는 제공되는 정보의 종류에 따라 2D Drawing, 3D Model, MEP(Mechanical, Electronical, and Plumbing) 기반의 연구로 구분된다\cite{ebbesen_information_2015}. 2D Drawing 정보 제공\cite{yeh_-site_2012,ishii_augmented_2002,cote_augmented_2013}은, 2D 도면 기반으로 실시간 도면 검색과 상세한 건축 정보를 제공하기 위한 연구이다. 이러한 연구는 기존 Offline 2D도면의 휴대성을 높이고, 빠르게 기호 및 정보를 검색해주는 기능을 제공하는 장점이 있으나, 3차원 모델에 대한 정보는 제공되지 않기 때문에 이에 대한 Image Mismatch의 위험이 계속 된다는 문제가 있다. 다음으로 3차원 모델 정보 제공은, 설계 단계에서 만들어진 2차원 도면 정보를 기반으로 3차원 모델에 대한 정보를 제공하는 연구이다\cite{wagner_building_2012,dong_collaborative_2013, hou_combining_2014,behzadan_ubiquitous_2008,williams_bim2mar:_2015}. 이러한 연구는 현실 환경과 같이 건축물의 3차원 정보를 조망할 수 있다는 장점이 있으나, 상세한 건축 정보를 획득하기에 어려움이 존재한다. 마지막으로 MEP 정보 제공 연구\cite{schall_handheld_2009,olbrich_augmented_2013,kwon_defect_2014,webster_augmented_????,golparvar-fard_d4ar4-dimensional_2009}는 BIM정보를 기반으로 건물 내의 구조물에 대한 정보를 제공하는 연구이다. 이는 3차원 조망이 가능하고, 상세한 건축 정보를 획득할 수 있다는 장점이 있으나, 세부적인 부분의 정보만 제공되므로 전체적인 구조를 파악하기 어려운 문제가 있다\cite{webster_augmented_????}. 이러한 연구들에서 보듯이 제공되는 정보는 다양한 정보를 제공할 수록 현장에서 정보의 획득이 쉽고, 의사 소통에 도움을 제공할 수 있다는 장점이 있다. 따라서 효율적인 정보의 제공을 위해서는 2D Drawing이나 3D 모델, MEP 정보의 전체적인 정보 제공이 필요하다. 

또 다른 문제는, 프로젝트에 참여한 다양한 관계자 사이의 의사소통이 어렵다는 점이다\cite{ishii_augmented_2002,klein_imaged-based_2012,chi_development_2012,lin_using_2014}.건축 프로젝트는 설계부터 시공과 관리까지 여러 단계(phase)에서 수많은 전문가들이 참여하기 때문에 효율적인 커뮤니케이션을 달성하기 매우 어렵다. 따라서 각각의 단계에 참여한 유사한 분야의 전문가들 사이의 의사소통\cite{ishii_augmented_2002,chi_development_2012,song_penlight:_2009,lin_using_2014}부터 단계 사이(between multiple disciplines)에서의 의사소통\cite{machino_remote-collaboration_2006,wagner_building_2012,lin_using_2014}까지 수많은 의사소통의 요구가 존재한다. %특히 실제 건축이 이루어지는 시공 현장에서는 자재 조달\cite{hollenbeck_multilevel_1995,lin_using_2014}, 설계 변경\cite{song_penlight:_2009}, 시공 안전\cite{bae_high-precision_2013} 등으로 인하여 많은 커뮤니케이션의 요구가 존재하고 있다. 
특히, 제공받은 정보에 대한 의사소통이 제대로 이루어지지 않을 경우, 비용과 건축 기간의 증가\cite{schall_handheld_2009}로 실제 결과물에 큰 차이가 날 수 있다. 

이러한 문제를 극복하기 위하여 건축 관계자들 간의 협업을 위해 다양한 연구들이 진행되고 있다. 먼저, Desktop 기반의 연구[3, 15, 21]들은 고정적인 환경인 Off-Site에서 진행되며, 전체적인 조망이나 정보 공유가 가능하다. 이러한 연구들의 경우, Off-Site 에서의 협업은 가능하나 시스템 구조상 On-Site에서는 활용되기 어렵다는 단점이 있다. 이러한 단점을 극복하기 위해 On-Site에서 동작 가능한 모바일 기반의 연구들[2, 11, 12, 17-19]이 진행되었다. 실제 건축 현장에서 Mobile Augmented Reality 기술을 이용하여 도면이나 3D 모델 등의 정보를 확인 할 수 있고, 필요에 따라 수정 가능하다. 하지만 모바일 기기의 특성상 화면 공유보다는 개인을 위한 Personal Workspace를 제공하기 때문에 협업에 제한적이다. 이러한 문제를 극복하기 위해, Desktop과  Mobile 기반 연구들의 단점을 보완하여 On-Site에서 건축 관계자들의 협업을 지원하기 위해 Projection 기반의 Interactive surface 기반의 연구들[4, 8, 10, 20]이 제시되고 있다. 이러한 기술은 Projection 기술을 이용하여 Interaction Space를 제공하기 때문에 여러 사용자들이 참여하여 협업하기에 적합하다. 따라서, 현장에서의 효율적인 의사소통을 위해서는, 개인화 된 기기 보다는 여러 사용자의 참여가 가능한 Projection 기반의 Interactive Surface 기술을 이용하는 것이 적합하다.

% 현대 건축 산업은 AEC/FM(Architecture, Engineering, Construction and Facility Management) 각 단계에서 다양한 전문가들이 참여하여, 분업과 협업을 통하여 프로젝트가 진행된다. 따라서, 이러한 전문가들이 적절하게 융합되어 프로젝트 팀을 구성하고\cite{hollenbeck_multilevel_1995}, 팀 내의 협업과 의사소통이 원활할 때 높은 품질의 건축 결과를 얻을 수 있다\cite{kwon_defect_2014,lin_using_2014,yeh_-site_2012}.

% 하지만, 건축 프로젝트는 여러 단계(phase)에서 수많은 전문가들이 참여하기 때문에 효율적인 커뮤니케이션을 달성하기 매우 어렵다\cite{ishii_augmented_2002,klein_imaged-based_2012,chi_development_2012,lin_using_2014}. 이는 건축 단계 사이(between multiple disciplines)에서의 의사소통의 어려움\cite{machino_remote-collaboration_2006,wagner_building_2012,lin_using_2014} 뿐만 아니라, 하나의 단계의 참여자들 사이에서의 의사소통\cite{ishii_augmented_2002,chi_development_2012,song_penlight:_2009,lin_using_2014}의 어려움이 지적되고 있다. 특히 실제 건축이 이루어지는 시공 현장에서는 자재 조달\cite{hollenbeck_multilevel_1995,lin_using_2014}, 설계 변경\cite{song_penlight:_2009}, 시공 안전\cite{bae_high-precision_2013} 등으로 인하여 많은 커뮤니케이션의 요구가 존재하고 있다.


% Offline 2D 도면은 건축 현장에서 사용하기에 이동성이 떨어지고\cite{yeh_-site_2012}, 2D 이미지에서 3D 모델을 유추해야 하기 때문에 인지적 부담이 있으며\cite{cote_augmented_2013,yeh_-site_2012}, 약속된 기호(predefined symbol)을 사용하여 이를 해석\cite{chi_research_2013}해야 하기 때문에 현장에서 활용이 떨어지는 문제점이 있다. 이로 인해 현장에서 정보 활용이 부족해지는 문제가 있다.


% 하지만, 이러한 시공 환경에서는 offline 2D 도면을 사용하여 커뮤니케이션하므로 정보 접근의 효율이 떨어지는 점이 가장 큰 문제로 지적되고 있다\cite{yeh_-site_2012,cote_augmented_2013,chi_research_2013}. Offline 2D 도면은 건축 현장에서 사용하기에 이동성이 떨어지고\cite{yeh_-site_2012}, 2D 이미지에서 3D 모델을 유추해야 하기 때문에 인지적 부담이 있으며\cite{cote_augmented_2013,yeh_-site_2012}, 약속된 기호(predefined symbol)을 사용하여 이를 해석\cite{chi_research_2013}해야 하기 때문에 현장에서 활용이 떨어지는 문제점이 있다. 이로 인해 현장에서 정보 활용이 부족해지는 문제가 있다.

% 이러한 문제점을 극복하고 효율적인 협업 환경을 지원하고, 현장에서 편리하게 정보를 제공하기 위하여 건축 정보의 Visualization에 대한 연구가 이루어지고 있다.





% 또한, 시스템적으로 가상현실 기술 기반, 모바일 어플리케이션 기반, 증강현실 기술 기반, Interactive Surface 기반의 연구들이 진행되고 있다. 가상현실 기술 기반의 연구들\cite{lin_using_2014,dong_collaborative_2013}들은 현장보다는 사전 준비 단계에서 활용되며, 전체적인 조망이 가능하다는 장점이 있으나, 시스템 구조 상 on-site에서는 활용되기 어렵다는 문제가 있다. xxx

%%%%%%


% 본 논문은 이러한 문제점을 극복하고 건축 현장에서 실제로 활용이 가능한 시스템을 제공하기 위하여 Interactive Surface 기술\cite{grossman__2010} 기반의 2D/3D 건축 정보 Visualization System을 제안한다. 특히, 건축 현장에서 활용도를 높이기 위하여 모바일 프로젝터와 깊이 인식 카메라를 이용하여 portable 시스템을 구현하였다. 기존 Interactive Surface 기술과 다르게 2D/3D 정보를 동시에 보여주기 위하여 Multiscreen Interactive Surface 기술을 적용하였으며, Portable 하게 Multiscreen을 구현하기 위하여 Image Marker 기반의 Calibration 기술을 개발하여 미리 설치된 환경이 아니라 실시간으로 Multiscreen을 구성하도록 하였다. 이를 이용하여 현장의 'L'-shape 벽면에 걸쳐 프로젝션함으로써 2D Interaction 을 위한 Horizontal Screen과 3D Interaction을 위한 Vertical Screen을 구성하였다. 이를 통하여 건설 현장에서도 쉽고 빠르게 건축 정보에 접근함으로써 현장에서의 협업을 높일 수 있었다. 

% 하면서 실시간으로 건축 모델의 정보를 업데이트하고, 작업 중에도 편리하게 협업할 수 있는 시스템을 설계하였다. 본 논문에서는 이러한 시스템을 이용하여 건축 현장에 적용하기 위한 design을 수행하고, 이를 건축 관련자들에게 사용 후 informal user study와 비교 연구를 적용하여 시스템의 유용성을 검증하였다. 추가적으로 실험의 피드백과 구현 상의 문제들을 정리하고 결론을 도출하였다.
% 논문의 구성은 2장에서 제안하는 시스템의 구조와 HW, SW 설계 및 구현, 상호작용 설계에 대하여 설명한다. 3장에서는 사용자 대상의 실험 설계 및 결과에 대하여 서술하고, 4장에서 실험 결과에 대한 discussion을 기술한다. 그리고 5장에서 결론을 짓도록 한다.

본 논문에서는 위에서 언급한 협업 문제와 효율적인 정보 제공 문제를 해결하고자 한다. 이를 위해, 건축 현장에서 활용 가능한 Portable projection 기반 Interactive Surface 기술을 이용하여 건축 정보 시스템을 구성하였다. 본 연구에서 제안하는 시스템은 효율적인 협업을 위해 프로젝터를 이용하여 Shared workspace을 제공하여 현장의 여러 사용자들이 참여가 가능하도록 설계하였고, 2D 도면의 정보와 3D 모델의 정보를 동시에 제공함으로써 정보 접근의 효율성을 높였다. 2D 도면정보와 3D 모델 정보를 동시에 제공하기 위하여, multi-screen 기반 interactive surface 기술\cite{coram_astrotouch:_2013,weiss_benddesk:_2010,wimmer_curve:_2010,benko_miragetable:_2012}을 구현하였다. multi-screen 기반 interactive surface기술의 stationary한 기존 기술의 한계를 극복하기 위하여 마커 기반의 실시간 calibration 기술을 개발하였다. 이렇게 구성된 Interactive Surface 환경에서 직관적인 건축 정보의 제어를 위해 Natural User Interface(NUI) 기술을 이용하여 상호작용 하도록 하였다. 이는 seamless하게 건축 정보에 접근하면서 실시간으로 건축 모델의 정보를 업데이트하며, 편리하게 조작이 가능하다. 
본 논문에서는 이러한 시스템을 건축 현장에 적용하고, 실제 건축 관계자들을 대상으로  informal user study와 비교 연구를 진행하여 시스템의 유용성을 검증하였다. 이러한 실험을 기반으로 피드백과 구현상의 보완을 진행하여 결론을 도출하였다.

논문의 구성은 2장에서 제안하는 시스템의 구조와 HW, SW 설계 및 구현, 상호작용 설계에 대하여 설명한다. 3장에서는 사용자 대상의 실험 설계 및 결과에 대하여 서술하고, 4장에서 실험 결과에 대한 discussion을 기술한다. 그리고 5장에서 결론을 짓도록 한다.



