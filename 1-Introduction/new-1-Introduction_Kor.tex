%!TEX root = ../Sensors_SmartConstruction.tex
% -*- root: ../Sensors_SmartConstruction.tex -*-

현대 건축 산업은 AEC/FM(Architecture, Engineering, Construction and Facility Management) 각 단계에서 다양한 전문가들이 참여하여, 분업과 협업을 통하여 프로젝트가 진행된다. 따라서, 이러한 전문가들이 적절하게 융합되어 프로젝트 팀을 구성하고\cite{hollenbeck_multilevel_1995}, 팀 내의 협업과 의사소통이 원활할 때 높은 품질의 건축 결과를 얻을 수 있다\cite{kwon_defect_2014,lin_using_2014,yeh_-site_2012}.

하지만, 건축 프로젝트는 여러 단계(phase)에서 수많은 전문가들이 참여하기 때문에 효율적인 커뮤니케이션을 달성하기 매우 어렵다\cite{ishii_augmented_2002,klein_imaged-based_2012,chi_development_2012,lin_using_2014}. 이는 건축 단계 사이(between multiple disciplines)에서의 의사소통의 어려움\cite{machino_remote-collaboration_2006,wagner_building_2012,lin_using_2014} 뿐만 아니라, 하나의 단계의 참여자들 사이에서의 의사소통\cite{ishii_augmented_2002,chi_development_2012,song_penlight:_2009,lin_using_2014}의 어려움이 지적되고 있다. 특히 실제 건축이 이루어지는 시공 현장에서는 자재 조달\cite{hollenbeck_multilevel_1995,lin_using_2014}, 설계 변경\cite{song_penlight:_2009}, 시공 안전\cite{bae_high-precision_2013} 등으로 인하여 많은 커뮤니케이션의 요구가 존재하고 있다.
하지만, 이러한 시공 환경에서는 offline 2D 도면을 사용하여 커뮤니케이션하므로 정보 접근의 효율이 떨어지는 점이 가장 큰 문제로 지적되고 있다\cite{yeh_-site_2012,cote_augmented_2013,chi_research_2013}. Offline 2D 도면은 건축 현장에서 사용하기에 이동성이 떨어지고\cite{yeh_-site_2012}, 2D 이미지에서 3D 모델을 유추해야 하기 때문에 인지적 부담이 있으며\cite{cote_augmented_2013,yeh_-site_2012}, 약속된 기호(predefined symbol)을 사용하여 이를 해석\cite{chi_research_2013}해야 하기 때문에 현장에서 활용이 떨어지는 문제점이 있다. 이로 인해 현장에서 정보 활용이 부족해지는 문제가 있다.

이러한 문제점을 극복하고 효율적인 협업 환경을 지원하고, 현장에서 편리하게 정보를 제공하기 위하여 건축 정보의 Visualization에 대한 연구가 이루어지고 있다. 아러한 연구는 제공되는 정보의 종류에 따라 2D Drawing, 3D Model, MEP(Mechanical, Electronical, and Plumbing) 기반의 연구로 구분된다\cite{ebbesen_information_2015}. 2D Drawing 정보 제공\cite{yeh_-site_2012,ishii_augmented_2002,cote_augmented_2013}은, 2D 도면 기반으로 실시간 도면 검색과 상세한 건축 정보를 제공하기 위한 연구이다. 이러한 연구는 기존 Offline 2D도면의 휴대성을 높이고, 빠르게 기호 및 정보를 검색해주는 기능을 제공하는 장점이 있으나, 3차원 모델에 대한 정보는 제공되지 않기 때문에 이에 대한 인지점 부담이 계속 된다는 문제가 있다. 다음으로 3차원 모델 정보 제공은, 설계 단계에서 만들어진 2차원 도면 정보를 기반으로 3차원 모델에 대한 정보를 제공하는 연구이다\cite{wagner_building_2012,dong_collaborative_2013, hou_combining_2014,behzadan_ubiquitous_2008}. 이러한 연구는 현실 환경과 같이 건축물의 3차원 정보를 조망할 수 있다는 장점이 있으나, 상세한 건축 정보를 획득하기에 어려움이 존재한다. 마지막으로 MEP 정보 제공 연구\cite{schall_handheld_2009,olbrich_augmented_2013,kwon_defect_2014}는 BIM정보를 기반으로 건물 내의 구조물에 대한 정보를 제공하는 연구이다. 이는 3차원 조망이 가능하고, 상세한 건축 정보를 획득할 수 있다는 장점이 있으나, xxx한 문제가 있다. 이러한 연구들에서 보듯이 제공되는 정보는 다양한 정보를 제공할 수록 현장에서 정보의 획득이 쉽고, 의사 소통에 도움을 제공할 수 있다는 장점이 있다. 따라서 2D Drawing이나 3D 모델, MEP 정보의 전체적인 정보 제공이 필요하다. 

또한, 시스템적으로 가상현실 기술 기반, 모바일 어플리케이션 기반, 증강현실 기술 기반, Interactive Surface 기반의 연구들이 진행되고 있다. 가상현실 기술 기반의 연구들\cite{lin_using_2014,dong_collaborative_2013}들은 현장보다는 사전 준비 단계에서 활용되며, 전체적인 조망이 가능하다는 장점이 있으나, 시스템 구조 상 on-site에서는 활용되기 어렵다는 문제가 있다. xxx

% 기존 건축 환경의 문제점

% 설계와 실제 건축물 사이의 차이로 인한 비용과 건축 기간 증가 (Kwon, et al., 2014), (Golparvar-Fard, et al, 2009)
% 건축 현장에서 관련자들 간의 커뮤니케이션이 어려움 (Ishii, et al., 2002), (Kim, et al., 2012), (Chi, et al., 2013) ((Olbrich et al., 2013)), (Ammari & Hammad, 2014)(Lin, Liu, Tsai, & Kang, 2014)

% 이를 극복하기 위한 건축 정보 Visualize (“Grand Challenges in Data and Information Visualization for the Architecture, Engineering, Construction, and Facility Management Industries,” n.d.)  (Etzold, Grimm, Schweitzer, & Dörner, 2014)


% ?와 Collaboration 시스템이 개발되고 있음
% Visualize 시스템: 건축 현장에서 기존 설계 도면 방식 정보 제공의 문제점을 극복하기 위해, 현장에서 필요한 정보를 편리한 형태로 제공하기 위한 기술
% 모바일 어플리케이션 기반 방법 (Saidi, et al, 2002), (Khoury & Kamat, 2009)
% 건축 현장에서 개인의 모바일 기기에서 BIM 정보 제공
% (Saidi, 2002): 개인 모바일 기기를 이용하여 현재 건축 진행 상황과 자제 정보 등을 제공
% (Khoury & Kamat, 2009): 상황 인지 기술을 이용하여 모바일 기기에 사용자 환경에 적합한 정보 Pushing
% 한계: 개인화 된 모바일 기기를 사용하기 때문에 협업이 어려움 (Yeh, 2012), (Ishii, et al., 2002)
% 모바일 증강현실 기반 방법 (Côté, Trudel, Snyder, & Gervais, 2013), (Williams, Gheisari, Chen, & Irizarry, 2015), (Schall et al., 2009)
% 모바일 증강현실 기술은 다양한 현장에서 활용이 가능하기 때문에 건축 환경에서 다양한 형태로 활용되고 있음
% (Williams, Gheisari, Chen, & Irizarry, 2015), (Behzadan, Aziz, Anumba, & Kamat, 2008), (Schall et al., 2009): GPS와 센서 기반의 증강현실 기술. 넓은 범위에서 활용 가능. 실내에서 활용이 어렵고 정밀한 정합 제공 어려움
% (Côté, Trudel, Snyder, & Gervais, 2013), , (Schall et al., 2009) (Bae, Golparvar-Fard, & White, 2013): 비전 기술을 이용한 증강현실 기술. 정밀한 정합 가능. 기준 위치 인지 어려움. 
% 한계: 개인화 된 모바일 기기를 사용하기 때문에 협업이 어려움 (Yeh, 2012), (Ishii et al., 2002), (Cote, 2013)


본 논문은 이러한 문제점을 극복하고 건축 현장에서 실제로 활용이 가능한 시스템을 제공하기 위하여 Interactive Surface 기술\cite{grossman__2010} 기반의 2D/3D 건축 정보 Visualization System을 제안한다. 특히, 건축 현장에서 활용도를 높이기 위하여 모바일 프로젝터와 깊이 인식 카메라를 이용하여 portable 시스템을 구현하였다. 기존 Interactive Surface 기술과 다르게 2D/3D 정보를 동시에 보여주기 위하여 Multiscreen Interactive Surface 기술을 적용하였으며, Portable 하게 Multiscreen을 구현하기 위하여 Image Marker 기반의 Calibration 기술을 개발하여 미리 설치된 환경이 아니라 실시간으로 Multiscreen을 구성하도록 하였다. 이를 이용하여 현장의 'L'-shape 벽면에 걸쳐 프로젝션함으로써 2D Interaction 을 위한 Horizontal Screen과 3D Interaction을 위한 Vertical Screen을 구성하였다. 이러한 환경에서 NUI 기술을 이용하여 직관적으로 3차원 정보를 획득하고 상호작용 하도록 하였다. 이를 통하여 건설 현장에서도 seamless하게 건축 정보를 접근하면서 실시간으로 건축 모델의 정보를 업데이트하고, 작업 중에도 편리하게 협업할 수 있는 시스템을 설계하였다. 본 논문에서는 이러한 시스템을 이용하여 건축 현장에 적용하기 위한 design을 수행하고, 이를 건축 관련자들에게 사용 후 informal user study와 비교 연구를 적용하여 시스템의 유용성을 검증하였다. 추가적으로 실험의 피드백과 구현 상의 문제들을 정리하고 결론을 도출하였다.
논문의 구성은 2장에서 제안하는 시스템의 구조와 HW, SW 설계 및 구현, 상호작용 설계에 대하여 설명한다. 3장에서는 사용자 대상의 실험 설계 및 결과에 대하여 서술하고, 4장에서 실험 결과에 대한 discussion을 기술한다. 그리고 5장에서 결론을 짓도록 한다.
