%!TEX root = ../construction.tex
% -*- root: ../construction.tex -*-
\begin{comment}
In the construction process, various stakeholders attend and cooperate with each other to move the project forward. However, it is currently difficult for architects' blueprints to be sufficiently adapted on a construction site, and there are many problems that have not been considered. It is also difficult to reflect modifications made on-site in overall models, and the difficulty in sharing this changed information between workers can be a problem. This system construction may be too complicated to be used on-site or 3D model information may not be offered, and since input technology is complicated, it is difficult to reflect this work information on-site. This study offers a 2D and 3D Dual Workspace using portable Projection AR to solve these problems, and 2D/3D Bimanual Interaction for pen and fingers was designed so that construction workers unused to IT can intuitively use the system. The system makes convenient use of overall construction model information using mobile computing technology within construction work, and modifications are quickly reflected in its model. Through this, it allows for easy communication. This system was designed to use interaction. It is designed for use by construction managers using an interactive design. Informal user studies were carried out for the proposed system comparing previous blueprint-based methods and task operation time to verify increased on-site usability. 
\end{comment}
% Editage 버전
Various stakeholders participate in construction and cooperate to move a project forward. However, it is difficult to adapt blueprints adequately on a construction site or to reflect modifications made on site in overall models, and sharing this changed information among workers. This study offers a 2D and 3D dual workspace using portable projection augmented reality to solve these problems and 2D/3D bimanual interaction for pen and fingers designed to enable construction workers unaccustomed to information technology to use the system intuitively. The system uses overall construction model information with mobile computing technology in construction work, with modifications being quickly reflected in its model, allowing for easy communication. This system was designed to be used by construction managers using an interactive design. Informal user studies were conducted for the proposed system comparing previous blueprint-based methods and task operation time to verify increased on-site usability.


%%%%%%%%%%%%%%%%%%%%%%%%%%%% UbiComp 버전
% 제임스 버전
\begin{comment}
Collaboration among stakeholders in the construction process is vital to completing projects successfully and on time. However, contruction engineers face the challenging problem of looking  at various architectural drawing perspectives. The perspectives can be complex, specific to a specialized job, and does not show a complete 3D view of the building. In this paper, we solve this problem by using Portable Projection AR technology to provide 2D and 3D dual workspace giving both a familar workspace feel and an intuitive system using pen and fingers for 2D/3D bimanual interaction. Based on feedback responses from construction workers who evaluated our prototype our system is shown to be more convenient and intuitive accessing information from construction drawings.
\end{comment}


% 번역소 버전
\begin{comment}
Construction processes are achieved by executing cooperation each other from diverse stakeholders' participation. But, at present construction sites, there is a problem that construction blueprints of architects are not approached and examined enough. This is because the system configuration is too complicated, 3D model information is not provided, or on-site work records are not reflected on the blueprints. To solve these problems, this paper provides 2D and 3D dual workspaces using projection AR technology. Also, in order that construction workers unfamiliar with IT technology can use systems intuitively, 2D/3D bimanual interaction of pens and fingers was designed. Through them, targeting real construction workers, from evaluating video prototypes, it was confirmed to access to construction blueprints and information more conveniently and intuitively on site.
\end{comment}
%%%%%%%%%%%%%%%%%%%%%%%%%%%% UbiComp 버전