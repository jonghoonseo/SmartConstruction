%!TEX root = ../construction.tex
% -*- root: ../construction.tex -*-

%\begin{comment}


%-----------------------------------------------------
\subsection{Computer Aided Construction}

\cite{harrison_architects_design_????}는 건축 공정을 예비설계, 건축 문서, 건축 관리의 3가지 단계로 구분하였다. IT 기술이 발전하면서 각 단계의 작업 효율성을 높이기 위한 다양한 연구들이 진행되고 있다.

첫번째 preliminary design phase는, architect가 자신의 conceptual design을 preliminary design으로 현실화(realize)하는 것을 도와주는 연구들이 진행되고 있다\cite{bae_ilovesketch:_2008, igarashi_teddy:_2007, song_modelcraft_2009, yu_prototype_2007, zeleznik_sketch:_2007}. 이러한 연구들은 기존 WIMP(Windows, Icons, Menus, Pointer)\cite{wikipedia_contributors_wimp_2014} 기반의 복잡한 인터페이스 대신 펜이나 손과 같은 Natural한 인터페이스를 사용한다. 이를 통하여 좀 더 빠르고 자연스럽게 프로토타이핑을 수행하여 architect가 drawing보다는 창의적인 컨셉 디자인에 집중할 수 있도록 한다. 

두번째 Construction documents를 위한 기술들은 Preliminary Design을 실제 건축에 사용될 수 있는 건축 문서로 변환하는 과정이다. 이러한 과정은 \textit{AutoCAD, 3dsMax, and SketchUp}과 같은 상용화 된 CAD software 들이 제공해주고 있다\cite{autodesk_3ds_????, autodesk_autocad_????, google_sketchup_????}.

세번째 Construction administration 영역의 연구는, 앞서 만들어진 건축 문서를 이용하여 건설 현장에 적용하는 과정을 보조하는 연구들이 진행되고 있다. 먼저 \cite{behzadan_enabling_2013, kim_interactive_2012} 연구들은 증강현실 기술을 이용하여 사전에 건축물이나 장비(equipments)들을 현장에 배치하여보고, 시뮬레이션을 제공하여 Conflict 등을 사전에 방지하는 연구들이 진행되었다. 이러한 연구들은 off-site에서 사전에 3D 객체를 manipulation하여 위험요소 등을 현실감 있게 고려(consider)할 수 있다는 장점이 있으나 실제 현장(in-site)에서 사용이 어렵다는 한계가 있다. 반면에 작업 현장에서 실시간으로 건축 정보에 접근하여 안전하고 정확한 시공을 제공하기 위한 연구들도 다양하게 진행되고 있다. \cite{aziz_supporting_2012, chen_framework_2011, giretti_design_2009}는 모바일 폰을 이용하여 필요한 정보를 즉시 접근할 수 있는 시스템 프레임워크를 설계하였다. 이러한 연구는 스마트 폰의 다양한 센서들을 Context Aware data로 사용하여, 건축 노동자가 필요한 정보를 적시에 제공할 수 있다. 하지만, 이러한 연구는 스마트 폰의 작은 인터페이스를 사용하기 때문에 3차원 데이터를 조작하기 어렵고, 양 손이 제한되며, 입력 인터페이스가 불편하여 현장에서으 피드백을 갱신하기 어렵다는 한계가 있었다. 이러한 시스템의 한계를 극복하기 위하여 \cite{behzadan_visualization_2005, khoury_high-precision_2009, yeh_-site_2012}는 Wearable Computing 기반의 방법으로 정보를 제공해주는 연구들이 진행되었다. Wearable Computing 기기를 사용하여 양 손이 자유로운 환경에서도 부가적인 정보를 얻을 수 있고 몰입도가 높아지는 장점이 있으나, 여전히 현장에서의 수정 사항 등을 반영하기 위한 인터페이스는 불편하거나 제공되지 않았다. Song, et al.\cite{song_penlight:_2009, song_mouselight:_2010}는 Interactive Surface 기술을 이용하여 입력 인터페이스 문제를 해결하였다. 이 연구에서는 Projector의 소형화를 예측하였고, 이를 통하여 펜에 부착 가능하거나 마우스 형태의 소형 프로젝터 컨셉을 제시하여 이러한 펜에 부착된 프로젝터를 이용하여 설계 도면의 정보를 획득하고 필요한 수정 사항 등을 갱신하도록 하였다. 하지만 이 연구는 2D surface에 국한되기 때문에 건축물의 3차원 정보를 획득하거나 접근하기가 어려웠다. 

%-----------------------------------------------------
\subsection{Interactive 3D Smart Surface}
Interactive Smart Surface 기술은 일반적으로 프로젝터와 카메라를 이용하여 일반적인 환경에 컴퓨터 기반의 Interactivity를 제공하는 기술이다\cite{kane_bonfire:_2009}. 프로젝터는 일상적인 환경에 Display 기능을 추가로 제공해주기 때문에 이러한 Interactive Smart Space를 구성하기에 적합하다. 이러한 연구는 Wellner's DigitalDesk\cite{wellner_digitaldesk_1991, wellner_interacting_1993}에서 시작하여 Augmented Surface\cite{rekimoto_augmented_1999}, EnhancedDesk\cite{koike_integrating_2001} 등의 다양한 연구가 진행되었다. 일반적으로 이러한 연구들은 고정형 프로젝터 환경에서 책이나 종이, 벽 등의 평면적인 대상들과 상호작용하는 연구들이 주로 이루어졌다. 

이후 Everywhere Display\cite{pinhanez_everywhere_2001, pinhanez_creating_2003, sukaviriya_portable_2004} 는 steerable projector를 설계하여 방 안의 여러 공간을 동적으로 사용할 수 있도록 제안되었다. 이러한 dynamic projection smart space 기술은 PlayAnywhere\cite{wilson_playanywhere:_2005} 시스템에서 모바일 형태로 발전되었다. 이 후 프로젝터 기기의 소형화로 handheld 크기의 프로젝터가 등장하면서 이러한 연구는 활성화되었다\cite{cao_interacting_2006, raskar_rfig_2004} 특히, PenLight\cite{song_penlight:_2009}는 펜에 부착 가능할 정도로 작은 프로젝터에 대한 비전을 제시하며 다양한 상호작용 방법을 제시하였다. 이러한 연구는 이 후에 다양한 펜 부착형 Smart Space 기술로 발전되었다. \cite{kim_ar_2013}는 이러한 portability를 이용하여 in-situ에서 designing을 위한 시나리오를 제안하였으며, \cite{kim_ar_2014}은 펜이 가장 많이 사용되는 학습 환경에 적용하여 학생들이 단지 펜만을 이용하여 Interactive Learning을 할 수 있는 시스템을 제안하였다. 이외에도 handheld 형태\cite{huber_lightbeam:_2012, kim_ilight:_2010}나 노트북에 embedded 형태\cite{kane_bonfire:_2009} 등 다양한 형태의 portable smart space 기술들이 등장하고 있다. 제안하는 Port3DAr 역시 건축 현장에서 사용하기 위하여 소형 portable 프로젝터를 사용하였으며, 편리한 거치를 위하여 tripod에 거치하는 형태로 설계하였다.

이러한 Smart Space 기술은 2D 평면에 한정된다는 한계가 존재한다. 최근의 여러 연구들은 이러한 제약을 벗어나 3차원 공간에서 Interactive Smart Space를 구현하기 위한 여러 연구들이 제안되고 있다\cite{grossman__2010}. 2차원 평면 위에서 3차원 컨텐츠를 제공하기 위하여 다양한 부가적인 기기를 이용하는 연구들이 진행되고 있으나 부피가 크고 복잡하다는 한계까 있으므로 건축현장에 적용하기에 어려움이 있다. 특히 Vertical Screen과 Horizontal Screen의 Dual Screen 기반의 시스템\cite{weiss_benddesk:_2010, coram_astrotouch:_2013, wimmer_curve:_2010, benko_miragetable:_2012}은 2차원 정보와 3차원 정보를 함께 제공할 수 있기 때문에 건축 현장에서 적용하기에 적합하다. 하지만 기존의 시스템은 vertical screen과 horizontal screen이 각각 독립적인 display로 구성되어 있었기에 부피가 크다는 문제가 있었으나, 제안하는 Port3DAr은 1대의 프로젝터 스크린을 'L'-shaped 공간에 프로젝션하여 공유하여 사용하기 때문에 portability를 높일 수 있었다.

최근의 Interactive Smart Space 기술은 \cite{jones_illumiroom:_2013, steimle_flexpad:_2013}와 같이 non-planar space에 projection되는 기술들이 연구되고 있다. 이러한 기술을 적용함으로써 복잡한 건축 환경에서도 적용할 수 있을 것으로 기대된다. 

%\end{comment}